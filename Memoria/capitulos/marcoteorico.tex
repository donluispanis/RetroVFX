%%%%%%%%%%%%%%%%%%%%%%%%%%%%%%%%%%%%%%%%%%%%%%%%%%%%%%%%%%%%%%%%%%%%%%%%
% Plantilla TFG/TFM
% Escuela Politécnica Superior de la Universidad de Alicante
% Realizado por: Jose Manuel Requena Plens
% Contacto: info@jmrplens.com / Telegram:@jmrplens
%%%%%%%%%%%%%%%%%%%%%%%%%%%%%%%%%%%%%%%%%%%%%%%%%%%%%%%%%%%%%%%%%%%%%%%%

\chapter{Estado del arte}

\section{La demoscene}

\subsection{Qué es la demoscene}

La \emph{demoscene} es una subcultura informática cuyo principal objetivo es la creación de demostraciones técnicas llamadas \emph{demoscenes}. Una \emph{demoscene} es un programa autocontenido y por norma general de peso ligero que intenta explotar al máximo el \emph{software} y el \emph{hardware} de la máquina que la ejecuta, con el fin de generar efectos visuales y sonoros. El objetivo de una \emph{demoscene} suele consistir en mostrar el ingenio y las habilidades del programador, así como tratar de impresionar al público.\\

Además, aunque el \emph{demoscening} en sí mismo no se puede considerar una forma de arte, sí que es cierto que muchas demos poseen un cierto componente artístico.\\ 

Se distinguen principalmente dos tipos de demo\footnote{\url{http://www.oldskool.org/demos/explained/demo_reference.html}}:

\begin{itemize}
  \item \textbf{Demo}: programa que genera gráficos y sonido en tiempo real. Suele tener una extensión superior a 5 minutos y normalmente no tienen límite de tamaño. Una demo suele ser creada por un grupo de personas que incluye al menos un programador, un diseñador gráfico y un músico. Las demos actuales suelen estar realizadas en 3D y cuentan con aceleración gráfica por hardware. Las demos más antiguas o realizadas para plataformas más antiguas (conocidas como "demos \emph{oldskool}") son procesadas de forma íntegra por la CPU (pues las plataformas para las que se desarrollan no poseen GPU) y suelen combinar ilusiones 3D con efectos gráficos en 2D.
  \item \textbf{Intro}: una demo de corta duración. Una intro suele ser temática (mientras que una demo suele compilar distintas escenas/temáticas). Además, las intros no suelen superar los 5 minutos de duración y su tamaño tiende a estar restringido. Las principales categorías de intros son 64K (65536 bytes), 4K (4096 bytes) y 1K (1024 bytes).
\end{itemize}

Existen otras categorías de demo, aunque son mucho menos comunes, como las \textbf{mega demos} (demos de gran duración/extensión, compuestas por múltiples partes) o las \textbf{dentros} (intros cuyo propósito es ofrecer un avance de una demo por llegar, todavía en desarrollo).\\

Además, existen muchas otras categorías derivadas o relacionadas con la \emph{demoscene}, como la creación de gráficos procedurales. Una de las subcategorías más populares dentro de esta son las \textbf{4K images}, imágenes complejas y de alta resolución generadas proceduralmente por programas de 4096 bytes. También es posible encontrar categorías similares relacionadas con la generación procedural de archivos de música o vídeo. Las mayores diferencias entre estas producciones y las demos son su tiempo de ejecución (no se ejecutan en tiempo real) y las técnicas que usan (al no ser el tiempo una limitación, pueden usar algoritmos computacionalmente más costosos, pero que generan resultados más complejos).

\subsection{Orígenes de la demoscene}

A principios de los años 80, con la popularización de los primeros ordenadores personales, la computación dejó de ser algo que sucedía en universidades para pasar a abrirse al gran mercado. Con ello, llegó también la distribución del software, aunque en aquella época no se producía por internet, si no tan sólo por medios físicos, como los disquetes. Estos programas venían con protecciones de copia por parte de los desarrolladores para evitar su distribución ilegal. Poco tardaron, no obstante, en aparecer los primeros \emph{crackers}, personas que se dedicaban a eliminar las protecciones de copia del software para su distribución gratuita. Esto llevó a la creación de una subcultura informática basada en el \emph{cracking} de videojuegos y otros tipos de software, al margen de la legalidad. Esto se hacía no solo con la intención de poder ditribuir el software de forma gratuita, si no que también suponía una fuente de diversión y competición para los \emph{crackers}\footnote{\url{https://web.archive.org/web/20170726063815/http://tomaes.32x.de/text/faq.php\#2.3}}.\\

Es por ello que los denominados \emph{crackers} empezaron a "firmar" el software que \emph{crackeaban} con seudónimos que aparecían en los menús o en las intros de los juegos. Con el tiempo, la competición y la ambición de los \emph{crackers} fue aumentando, y llegó un punto en el que no solo se limitaban a quitar las protecciones de copia del software, si no que también creaban sus propias intros para los programas.\\

Es en este punto cuando la \emph{demoscene} empieza a tomar forma, cuando una parte de los \emph{crackers} deciden retornar a la legalidad pero sin dejar atrás la competición y la diversión. De este modo, este nuevo sector se empieza a dedicar a la creación de intros y demos cuyo objetivo es mostrar sus habilidades al resto de \emph{demosceners}\footnote{\url{http://widerscreen.fi/assets/reunanen-wider-1-2-2014.pdf}}.

\subsection{Composición y cultura de la demoscene}

La \emph{demoscene} es una subcultura informática muy centrada en el trabajo en equipo y en compartir. Con el desarrollo y popularización de la \emph{demoscene}, a partir de principios de los 90 se popularizó y se estandarizó, con la creación de eventos y competiciones.

Heredado de las \emph{copyparties}, eventos en los que \emph{crackers} y \emph{demosceners} se juntaban para conocerse y compartir software, al margen de la legalidad, nuevos eventos empezaron a crearse a principios de los noventa. Estos eventos sí eran legales y se centraban únicamente en el aspecto de las demos. Pasan a ser eventos sociales en los que los \emph{demosceners} se conocen, comparten y compiten.

Estos eventos, conocidos como \emph{demoparties}, pasan entonces a ser concursos y tener distintas categorías y premios. Para concursar, normalmente los competidores se juntaban en grupos, usualmente compuestos por al menos un programador, un diseñador gráfico y un músico. Estos grupos tenían su propio nombre e identidad. A su vez, cada uno de los componentes del grupo también solía usar un seudónimo. El uso de un seudónimo es una herencia de los orígenes de la \emph{demoscene} en el \emph{cracking}, aunque el propósito de usar un alias cambia. Mientras que los \emph{crackers} usaban un nombre falso para ocultar su identidad, pues realizaban actividades ilegales, los \emph{demosceners} usan este alias como una forma de expresión.

\subsection{La demoscene en la actualidad}

Si bien la \emph{demoscene} siempre se ha mantenido como una subcultura y nunca ha llegado a tener una popularidad masiva, su auge se dio en los años noventa.  En la actualidad, muchos de los eventos de \emph{demoscening} que se crearon en los 90 han desaparecido, y otros tantos han derivado en eventos dedicados a los ordenadores de forma mucho más genérica, derivando en eventos de software o en \emph{LAN-parties}.\\

Del mismo modo que muchas ferias y eventos han desaparecido, muchos otros también se han ido creando. No obstante, parece que hay una tendencia general hacia el olvido.\\

Las \emph{demoparties} en la actualidad suelen ser eventos locales, normalmente humildes, donde participan apasionados y nostálgicos.

(acabado ?) (hablar un poco más sobre qué cosas hay ahora (?), hablar sobre el rechazo por parte de algunos sectores a la inclusión del público joven, a que el hecho de ser un circulo muy cerrado puede ser causa de su extinción)

\section{Eventos de demoscening}

Principales eventos de demoscene

\section{Grupos de demoscening}

Grupos de demoscene, principales demos

\section{Demoscenes célebres}


\section{Influencia de la demoscene en la industria}

Importantes para la historia:
%http://widerscreen.fi/assets/reunanen-wider-1-2-2014.pdf
%http://www.oldskool.org/demos/explained/demo_history.html
%https://web.archive.org/web/20170726063815/http://tomaes.32x.de/text/faq.php#2.3.

Elevated y otras demos por el estilo

%https://web.archive.org/web/20170726063815/http://tomaes.32x.de/text/faq.php
%http://www.oldskool.org/demos/explained/demo_history.html

%http://www.oldskool.org/demos/explained/demo_reference.html

%http://www.demoscene.info
%http://www.pouet.net/index.php
%https://en.wikipedia.org/wiki/Assembly_(demoparty)
%https://en.wikipedia.org/wiki/Demoscene

%http://widerscreen.fi/numerot/2014-1-2/crackers-became-us-demosceners/
%http://www.oldskool.org/demos/explained/demo_pages.html
%ftp://ftp.hornet.org/pub/demos/info/demonews/1998/demonews.150