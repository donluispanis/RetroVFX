%%%%%%%%%%%%%%%%%%%%%%%%%%%%%%%%%%%%%%%%%%%%%%%%%%%%%%%%%%%%%%%%%%%%%%%%
% Plantilla TFG/TFM
% Escuela Politécnica Superior de la Universidad de Alicante
% Realizado por: Jose Manuel Requena Plens
% Contacto: info@jmrplens.com / Telegram:@jmrplens
%%%%%%%%%%%%%%%%%%%%%%%%%%%%%%%%%%%%%%%%%%%%%%%%%%%%%%%%%%%%%%%%%%%%%%%%

\chapter{Objetivos}

A continuación se listan los objetivos principales del siguiente trabajo:

\begin{itemize}
	\item \textbf{Entender la \emph{demoscene}}: entender y exponer la subcultura informática de la \emph{demoscene}, sus orígenes y motivación, los rasgos culturales compartidos por sus participantes, sus principales grupos y eventos, así como su legado.
	\item \textbf{Entender la importancia de conocer el bajo nivel}: qué es el bajo nivel y su relevancia en el contexto actual. Cómo influye el conocimiento del bajo nivel y del hardware en el rendimiento de un programa informático.
	\item \textbf{Crear pruebas de rendimiento}: realización de pruebas con el fin de conocer qué operaciones son más costosas de realizar en un computador y por qué. Exponer posibles mejoras y optimizaciones en el código para mejorar el rendimiento de un programa.
	\item \textbf{Entender e implementar los efectos más usados por la \emph{demoscene}}: recopilar las técnicas gráficas más comúnmente usadas en el origen de la \emph{demoscene}. Ofrecer una comprensión profunda y detallada de las mismas, desde un punto de vista tanto teórico como práctico. 
	\item \textbf{Crear una breve \emph{demoscene} original}: entender y compilar todo el conocimiento aprendido en una sola demo, que combine todos los efectos y mejoras de rendimiento estudiadas.
\end{itemize}