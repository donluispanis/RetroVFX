\chapter{Tests de rendimiento}

\section{Planteamiento inicial}

Siguiendo el carácter definido previamente, el objetivo de este trabajo es poder hacerlo todo desde cero, en la medida de lo que resulta posible o razonable.\\

Es por ello, que siguiendo este mismo espíritu busqué crear mi propio marco de trabajo para ejecutar pequeños tests de rendimiento. Esto es todo lo que debería ser capaz de hacer mi herramienta:

\begin{itemize}
	\item Ser capaz de ejecutar un test provisto de forma externa
	\item Ejecutar cada test múltiples veces, para poder obtener un abanico de tiempos de ejecución
	\item Extraer una media de los resultados obtenidos en la medición
	\item Mostrar los resultados obtenidos por terminal, así como almacenarlos en un fichero de texto
\end{itemize}

Al ser pocos los requerimientos de mi herramienta, el desarrollo de la misma debería ser breve y sencillo.\\

Además, llegó también el momento de plantear el carácter los tests a realizar:

\begin{itemize}
	\item Medir el coste de la suma, resta, multiplicación, división y módulo para enteros de 32 bits
	\item Medir el coste de la suma, resta, multiplicación, división y módulo para enteros de 64 bits
	\item Medir el coste de la suma, resta, multiplicación y división para números en coma flotante de 32 bits
	\item Medir el coste de la suma, resta, multiplicación y división para números en coma flotante de 64 bits
	\item Medir el coste de llamar a una función generadora de números aleatorios, llamar una función de coseno, de arco coseno, de raíz cuadrada y de raíz cúbica en comparación al coste de llamar a una función vacía
	\item Medir el coste de acceso aleatorio a memoria
	\item Medir el coste de acceso directo o indirecto a memoria
\end{itemize}

\section{Implementación}

Para inicializar la clase que manejará nuestros tests, debemos crearla pasándole por parámetro una cadena de caracteres que represente la ruta del fichero donde queremos que se guarde el resultado de nuestros tests. Tras mostrar por terminal un código de "bienvenida", abrimos el fichero con la ruta especificada y, a continuación, inicializamos dos \emph{buffers}. Será redirigiendo la salida de \lstinline{std::cout}\{ a cada uno de estos \emph{buffers} como conseguiremos escribir tanto en terminal como en fichero.

\begin{lstlisting}[style=C-color, caption={Constructor de nuestra clase para manejar tests}, label=cod:testtemplate, escapechar=|]
TestTemplate::TestTemplate(const char *logPath)
{
    std::cout << std::endl;
    std::cout << "Running tests... this may take a while" << std::endl;
    std::cout << "Output will be stored in " << logPath << std::endl;
    std::cout << std::endl;

    file = std::ofstream(logPath);
    writeToScreen = std::cout.rdbuf();
    writeToFile = file.rdbuf();
}
\end{lstlisting}

A continuación, llega el momento de crear un método que nos permita ejecutar nuestros tests, como podemos ver en el código [\ref{cod:executetest}]. Este método recibe una función por parámetro, que ejecutará internamente 100 veces el test pasado,  y almacenará el tiempo de cada ejecución en un vector.

\begin{lstlisting}[style=C-color, caption={Método para ejecutar un test}, label=cod:executetest, escapechar=|]
void TestTemplate::ExecuteTest(std::function<void(void)> test)
{
    timeCounts.clear();
    std::chrono::duration<double> elapsedSeconds;

    for (int i = 0; i < 100; i++)
    {
        time_point = std::chrono::system_clock::now();

        test();

        elapsedSeconds = std::chrono::system_clock::now() - time_point;
        timeCounts.push_back(elapsedSeconds.count());
    }
}
\end{lstlisting}

Será entonces, tras haber ejecutado nuestro test, que podremos obtener el tiempo medio de ejecución con el siguiente código:

\begin{lstlisting}[style=C-color, caption={Método para calcular la media del tiempo transcurrido}, label=cod:calculatetime, escapechar=|]
double TestTemplate::CalculateAverageTime()
{
    double time = std::accumulate(timeCounts.begin(), timeCounts.end(), 0.0) / (double)timeCounts.size();
    return time;
}
\end{lstlisting}

Proveeremos además con un método [\ref{cod:writeresults}] que dado el nombre del test, una descripción del mismo y el tiempo que ha tardado en ejecutarse, escribe de forma fácilmente legible para el ojo humano los resultados de la ejecución. El resultado se escribe tanto en fichero como por terminal, redirigiendo la salida de texto.

\begin{lstlisting}[style=C-color, caption={Método para calcular la media del tiempo transcurrido}, label=cod:writeresults]
void TestTemplate::WriteTestResultsIntoScreenAndFile(const char *testName, const char *testDescription, double testTime)
{
    auto output = [](const char *testName, const char *testDescription, double testTime) {
        std::cout << "| Test " << testName << ": " << testDescription << std::endl;
        std::cout << "|          Time spent: " << testTime << " seconds" << std::endl;
        std::cout << "|" << std::endl;
    };

    std::cout.rdbuf(writeToFile);
    output(testName, testDescription, testTime);

    std::cout.rdbuf(writeToScreen);
    output(testName, testDescription, testTime);
}
\end{lstlisting}

Tal vez lo mostrado hasta ahora pueda parecer un poco confuso, pero es al combinar todo este código para ejecutar un test cuando se ve la potencia de la clase que hemos creado, que nos permite manejar tests con facilidad, de forma legible y sencilla:

\begin{lstlisting}[style=C-color, caption={Ejemplo de uso de test template}, label=cod:example]
TestTemplate T("PerformanceResults.txt");

T.ExecuteTest([]() {for (int j = 0; j < 10000000; j++); });
T.WriteTestResultsIntoScreenAndFile("Loops", "Void For Loop executed 10000000 times", T.CalculateAverageTime());
\end{lstlisting}

\begin{lstlisting}[caption={Resultado del test}, label=cod:exampleresult]
| Test Loops: Void For Loop executed 10000000 times
| Time spent: 0.0168712 seconds
|
\end{lstlisting}

El test de ejemplo mostrado en el código [\ref{cod:example}] es bastante trivial, pues es un código que se limita a ejecutar un bucle. No obstante, se ejemplifica lo sencillo y limpio que se vuelve escribir un test, con un resultado también fácil de leer e interpretar [\ref{cod:exampleresult}].

\section{Resultados}

El resultado de cada test que se muestra a continuación ha sido hallado a partir de la media de 5 ejecuciones individuales de cada conjunto de tests para cada plataforma. A su vez, cada conjunto de tests ha sido ejecutado internamente un total de 100 veces, como podemos ver en [\ref{cod:executetest}], habiéndose hallado la media de tiempo de estas ejecuciones.\\

Los resultados que se muestran en la columna izquierda vienen dados en segundos, siendo el tiempo de ejecución medio del test. Los resultados que se muestran a la derecha son el resultado ponderado con respecto a un resultado base (por ejemplo, cúantas veces es más costosa una división que una suma).\\

Cada conjunto de tests ha sido compilado a su vez con tres objetivos distintos, siendo estos:

\begin{itemize}
	\item Ejecutable de \textbf{Windows x86} compilado con \textbf{g++ 7.4.0} para \textbf{MinGW32}
	\item Ejecutable de \textbf{Windows x64} compilado con \textbf{g++ 8.2.1} para \textbf{MinGW64}
	\item Ejecutable de \textbf{Linux x64} compilado con \textbf{g++ 8.2.1}
\end{itemize}

Los tests han sido ejecutados en los siguientes sistemas operativos:

\begin{itemize}
	\item \textbf{Windows 10 Home x64} (versión 1803)
	\item \textbf{Manjaro Linux x6}4 (versión 18.0.4)
\end{itemize}

Los tests han sido ejecutados bajo el siguiente hardware:

\begin{itemize}
	\item CPU: \textbf{Intel Core i7-6700HQ} 2.60GHz, x64 processor
	\item RAM: \textbf{16 GB DDR4-2133}
\end{itemize}

Cada operación matemática que se muestra a continuación se ha realizado dentro de un bucle que se ejecuta 10000000 millones de veces, por lo que el resultado que vemos no se debe usar tanto de forma cuantitativa como para establece comparaciones o relaciones.\\

En la tabla [\ref{tab:int32}] se muestran los resultados de realizar operaciones matemáticas con números enteros de 32 bits. En la columna de resultados relativos, se compara el coste de las operaciones con respecto al coste de la suma de enteros de 32 bits.\\

\begin{table}[h]
\centering
\begin{tabular}{@{}l|rrr|rrr|@{}}
\cmidrule(l){2-7}
                                     & \multicolumn{3}{c|}{Results (s)}                                                               & \multicolumn{3}{c|}{Relative results}                                                      \\ \cmidrule(l){2-7} 
                                     & \multicolumn{1}{c|}{Win x86} & \multicolumn{1}{c|}{Win x64} & \multicolumn{1}{c|}{Lin x64} & \multicolumn{1}{c|}{Win x86} & \multicolumn{1}{c|}{Win x64} & \multicolumn{1}{c|}{Lin x64} \\ \midrule
\multicolumn{1}{|l|}{Sum}            & 0,0232331                    & 0,0286147                    & 0,0284548                    & 0,0                          & 0,0                          & 0,0                          \\ \cmidrule(r){1-1}
\multicolumn{1}{|l|}{Subtraction}    & 0,0232895                    & 0,0268971                    & 0,0278414                    & 0,0                          & -0,2                         & -0,1                         \\ \cmidrule(r){1-1}
\multicolumn{1}{|l|}{Multiplication} & 0,0236683                    & 0,0272116                    & 0,0274386                    & 0,1                          & -0,1                         & -0,1                         \\ \cmidrule(r){1-1}
\multicolumn{1}{|l|}{Division}       & 0,0353099                    & 0,0400042                    & 0,0391396                    & 2,7                          & 1,1                          & 1,0                          \\ \cmidrule(r){1-1}
\multicolumn{1}{|l|}{Modulo}         & 0,0361311                    & 0,0392958                    & 0,0451121                    & 2,9                          & 1,0                          & 1,6                          \\ \bottomrule
\end{tabular}
\caption{Operaciones con enteros de 32 bits}
\label{tab:int32}
\end{table}

En la tabla [\ref{tab:int64}] se muestran los resultados de realizar operaciones matemáticas con números enteros de 64 bits. En la columna de resultados relativos, se compara el coste de las operaciones respecto al coste de la suma de enteros de 32 bits.\\

\begin{table}[h]
\centering
\begin{tabular}{@{}l|rrr|rrr|@{}}
\cmidrule(l){2-7}
                                     & \multicolumn{3}{c|}{Results (s)}                                                               & \multicolumn{3}{c|}{Relative results}                                                      \\ \cmidrule(l){2-7} 
                                     & \multicolumn{1}{c|}{Win x86} & \multicolumn{1}{c|}{Win x64} & \multicolumn{1}{c|}{Lin x64} & \multicolumn{1}{c|}{Win x86} & \multicolumn{1}{c|}{Win x64} & \multicolumn{1}{c|}{Lin x64} \\ \midrule
\multicolumn{1}{|l|}{Sum}            & 0,0233056                    & 0,0234745                    & 0,0241097                    & 0,0                          & -0,5                         & -0,4                         \\ \cmidrule(r){1-1}
\multicolumn{1}{|l|}{Subtraction}    & 0,0234734                    & 0,0238337                    & 0,0256926                    & 0,1                          & -0,4                         & -0,3                         \\ \cmidrule(r){1-1}
\multicolumn{1}{|l|}{Multiplication} & 0,0240932                    & 0,0245130                    & 0,0265930                    & 0,2                          & -0,4                         & -0,2                         \\ \cmidrule(r){1-1}
\multicolumn{1}{|l|}{Division}       & 0,0457833                    & 0,0462163                    & 0,1060608                    & 5,1                          & 1,6                          & 7,6                          \\ \cmidrule(r){1-1}
\multicolumn{1}{|l|}{Modulo}         & 0,0392737                    & 0,0393487                    & 0,1016039                    & 3,6                          & 1,0                          & 7,2                          \\ \bottomrule
\end{tabular}
\caption{Operaciones con enteros de 64 bits}
\label{tab:int64}
\end{table}

En la tabla [\ref{tab:float32}] se muestran los resultados de realizar operaciones matemáticas con números en coma flotante de 32 bits. En la columna de resultados relativos, se compara el coste de las operaciones respecto al coste de la suma de enteros de 32 bits.\\

\begin{table}[h]
\centering
\begin{tabular}{@{}l|rrr|rrr|@{}}
\cmidrule(l){2-7}
                                     & \multicolumn{3}{c|}{Results (s)}                                                               & \multicolumn{3}{c|}{Relative results}                                                      \\ \cmidrule(l){2-7} 
                                     & \multicolumn{1}{c|}{Win x86} & \multicolumn{1}{c|}{Win x64} & \multicolumn{1}{c|}{Lin x64} & \multicolumn{1}{c|}{Win x86} & \multicolumn{1}{c|}{Win x64} & \multicolumn{1}{c|}{Lin x64} \\ \midrule
\multicolumn{1}{|l|}{Sum}            & 0,0235983                    & 0,0247058                    & 0,0243978                    & 0,1                          & -0,4                         & -0,4                         \\ \cmidrule(r){1-1}
\multicolumn{1}{|l|}{Subtraction}    & 0,0236995                    & 0,0245818                    & 0,0247821                    & 0,1                          & -0,4                         & -0,4                         \\ \cmidrule(r){1-1}
\multicolumn{1}{|l|}{Multiplication} & 1,0653020                    & 0,0246152                    & 0,0242643                    & 233,6                        & -0,4                         & -0,4                         \\ \cmidrule(r){1-1}
\multicolumn{1}{|l|}{Division}       & 1,2454660                    & 0,0248546                    & 0,0251386                    & 273,9                        & -0,3                         & -0,3                         \\ \bottomrule
\end{tabular}
\caption{Operaciones en coma flotante con 32 bits}
\label{tab:float32}
\end{table}

En la tabla [\ref{tab:float64}] se muestran los resultados de realizar operaciones matemáticas con números en coma flotante de 64 bits. En la columna de resultados relativos, se compara el coste de las operaciones respecto al coste de la suma de enteros de 32 bits.\\

\begin{table}[h]
\centering
\begin{tabular}{@{}l|rrr|rrr|@{}}
\cmidrule(l){2-7}
                                     & \multicolumn{3}{c|}{Results (s)}                                                               & \multicolumn{3}{c|}{Relative results}                                                      \\ \cmidrule(l){2-7} 
                                     & \multicolumn{1}{c|}{Win x86} & \multicolumn{1}{c|}{Win x64} & \multicolumn{1}{c|}{Lin x64} & \multicolumn{1}{c|}{Win x86} & \multicolumn{1}{c|}{Win x64} & \multicolumn{1}{c|}{Lin x64} \\ \midrule
\multicolumn{1}{|l|}{Sum}            & 0,0240914                    & 0,0250733                    & 0,0253800                    & 0,2                          & -0,3                         & -0,3                         \\ \cmidrule(r){1-1}
\multicolumn{1}{|l|}{Subtraction}    & 0,0239746                    & 0,0246671                    & 0,0253605                    & 0,2                          & -0,4                         & -0,3                         \\ \cmidrule(r){1-1}
\multicolumn{1}{|l|}{Multiplication} & 0,2191568                    & 0,0249607                    & 0,0251803                    & 43,9                         & -0,3                         & -0,3                         \\ \cmidrule(r){1-1}
\multicolumn{1}{|l|}{Division}       & 1,2480080                    & 0,0256489                    & 0,0261598                    & 274,5                        & -0,3                         & -0,2                         \\ \bottomrule
\end{tabular}
\caption{Operaciones en coma flotante con 64 bits}
\label{tab:float64}
\end{table}

En la tabla [\ref{tab:functions}] se muestran los resultados de llamar a una función vacía, una función de generación de números aleatorios, una función de coseno, una función de arco coseno, una función de raíz cuadrada y una función de raíz cúbica. En la columna de resultados relativos, se compara el coste de las funciones respecto al coste de la suma de enteros de 32 bits.\\

\begin{table}[h]
\centering
\begin{tabular}{@{}l|rrr|rrr|@{}}
\cmidrule(l){2-7}
                             & \multicolumn{3}{c|}{Results (s)}                                                               & \multicolumn{3}{c|}{Relative results}                                                      \\ \cmidrule(l){2-7} 
                             & \multicolumn{1}{c|}{Win x86} & \multicolumn{1}{c|}{Win x64} & \multicolumn{1}{c|}{Lin x64} & \multicolumn{1}{c|}{Win x86} & \multicolumn{1}{c|}{Win x64} & \multicolumn{1}{c|}{Lin x64} \\ \midrule
\multicolumn{1}{|l|}{empty}  & 0,0244699                    & 0,0249656                    & 0,0242798                    & 0,3                          & -0,3                         & -0,4                         \\ \cmidrule(r){1-1}
\multicolumn{1}{|l|}{rand()} & 0,1180654                    & 0,1169742                    & 0,0821240                    & 21,3                         & 8,2                          & 5,3                          \\ \cmidrule(r){1-1}
\multicolumn{1}{|l|}{cos()}  & 0,6730564                    & 0,4900506                    & 0,2550030                    & 145,6                        & 42,7                         & 22,2                         \\ \cmidrule(r){1-1}
\multicolumn{1}{|l|}{acos()} & 2,7986080                    & 0,3835644                    & 0,0584450                    & 622,0                        & 32,8                         & 2,9                          \\ \cmidrule(r){1-1}
\multicolumn{1}{|l|}{sqrt()} & 0,3691394                    & 0,1106834                    & 0,0743690                    & 77,5                         & 7,6                          & 4,5                          \\ \cmidrule(r){1-1}
\multicolumn{1}{|l|}{cbrt()} & 0,8855260                    & 0,5628924                    & 0,3532132                    & 193,3                        & 49,4                         & 31,8                         \\ \bottomrule
\end{tabular}
\caption{Llamadas a funciones matemáticas}
\label{tab:functions}
\end{table}

En la tabla [\ref{tab:memory}] se muestran los resultados de realizar un millón de accesos a memoria, en el primer lado siendo memoria contigua y en los siguientes casos accediendo a memoria aumentando la distancia entre accesos en potencias de 10. En la columna de resultados relativos, se compara el coste de los accesos con el coste de acceder a memoria contigua.\\

\begin{table}[h]
\centering
\begin{tabular}{@{}l|rrr|rrr|@{}}
\cmidrule(l){2-7}
                                    & \multicolumn{3}{c|}{Results (s)}                                                               & \multicolumn{3}{c|}{Relative results}                                                      \\ \cmidrule(l){2-7} 
                                    & \multicolumn{1}{c|}{Win x86} & \multicolumn{1}{c|}{Win x64} & \multicolumn{1}{c|}{Lin x64} & \multicolumn{1}{c|}{Win x86} & \multicolumn{1}{c|}{Win x64} & \multicolumn{1}{c|}{Lin x64} \\ \midrule
\multicolumn{1}{|l|}{Lineal access} & 0,0025534                    & 0,0024107                    & 0,0023643                    & 0,0                          & 0,0                          & 0,0                          \\ \cmidrule(r){1-1}
\multicolumn{1}{|l|}{Jumps 10}      & 0,0039727                    & 0,0039288                    & 0,0038412                    & 0,6                          & 0,6                          & 0,6                          \\ \cmidrule(r){1-1}
\multicolumn{1}{|l|}{Jumps 100}     & 0,0060442                    & 0,0061442                    & 0,0059900                    & 1,4                          & 1,5                          & 1,5                          \\ \cmidrule(r){1-1}
\multicolumn{1}{|l|}{Jumps 1000}    & 0,0063647                    & 0,0064300                    & 0,0058503                    & 1,5                          & 1,7                          & 1,5                          \\ \cmidrule(r){1-1}
\multicolumn{1}{|l|}{Jumps 10000}   & 0,0095576                    & 0,0096220                    & 0,0056557                    & 2,7                          & 3,0                          & 1,4                          \\ \bottomrule
\end{tabular}
\caption{Accesos a memoria}
\label{tab:memory}
\end{table}

En la tabla [\ref{tab:alignment}] se muestran los resultados de acceder 10000000 a un elemento de una variable, siendo el primer acceso directo y el segundo indirecto (a través de un puntero). En la columna de resultados relativos, se compara el coste del acceso directo con el acceso indirecto.\\

\begin{table}[h]
\centering
\begin{tabular}{@{}l|crr|crr|@{}}
\cmidrule(l){2-7}
                                      & \multicolumn{3}{c|}{Results (s)}                                                                & \multicolumn{3}{c|}{Relative results}                                                      \\ \cmidrule(l){2-7} 
                                      & \multicolumn{1}{c|}{Win x86}  & \multicolumn{1}{c|}{Win x64} & \multicolumn{1}{c|}{Lin x64} & \multicolumn{1}{c|}{Win x86} & \multicolumn{1}{c|}{Win x64} & \multicolumn{1}{c|}{Lin x64} \\ \midrule
\multicolumn{1}{|l|}{Direct access}   & \multicolumn{1}{r}{0,0166824} & 0,0169107                    & 0,0156102                    & \multicolumn{1}{r}{0,0}      & 0,0                          & 0,0                          \\ \cmidrule(r){1-1}
\multicolumn{1}{|l|}{Indirect access} & \multicolumn{1}{r}{0,0220008} & 0,0223851                    & 0,0208514                    & \multicolumn{1}{r}{0,3}      & 0,3                          & 0,3                          \\ \bottomrule
\end{tabular}
\caption{Acceso a memoria directo e indirecto}
\label{tab:alignment}
\end{table}

\section{Análisis de los resultados}

A partir de los resultados obtenidos, se pueden extraer las siguientes conclusiones:

\begin{itemize}
	\item \textbf{Ejecutar código compilado para una máquina de 32 bits bajo una arquitectura de 64 bits puede tener penalizaciones severas}: como podemos ver, la mayoría de las operaciones corren más lento para el código compilado en 32 bits, siendo especialmente notorio en multiplicaciones y divisiones en coma flotante, así como en todas las operaciones matemáticas. Estas diferencias se deben principalmente al código ensamblador que se genera. Por ejemplo, para realizar una multiplicación en coma flotante para un ejecutable x32, se usa la instrucción \lstinline{fmuls}\{, mientras que para realizar una multiplicación para un ejecutable x64, se utiliza la instrucción \lstinline{mulss}\{. Cada una de estas instrucciones viene de un set de instrucciones y de una unidad distinta en la CPU, viniendo la primera de la unidad de coma flotante (FPU) mientras que la segunda viene del conjunto de instrucciones de extensión de SIMD (SSE). Las instrucciones SIMD (\emph{Single Instruction, Multiple Data}) son instrucciones que, como su nombre indica, se pueden ejecutar sobre múltiples datos de forma paralela, de modo que son mucho más rápidas que las que se ejecutan en la FPU. No obstante, las instrucciones SSE tienen precisión fija, mientras que las instrucciones para la FPU son más flexibles al poder tener precisión variable.
	\item \textbf{El sistema operativo es un factor determinante}: si comparamos los resultados entre las ejecuciones de los tests en x64, veremos que en general los resultados entre Linux y Windows son consistentes (como cabría esperar). No obstante, podemos ver algunas diferencias significativas, por ejemplo, en la tabla [\ref{tab:int64}], la división de enteros de 64 bits toma mucho más tiempo en Linux que en Windows. Se da el caso contrario en la tabla [\ref{tab:functions}], dónde todas las funciones resultan significativamente más rápidas en Linux. Ambos tests han sido ejecutados bajo la misma versión del compilador, aunque compilando en distintos sistemas operativos. Como podemos ver, los detalles de implementación son muy relevantes, así como la interacción con el sistema operativo.
	\item \textbf{La suma, multiplicación y división de enteros tiene un coste similar, la división y el módulo son al menos el doble de costosas}: como podemos ver en las tablas [\ref{tab:int32}] y [\ref{tab:int64}], no existe gran diferencia a nivel de coste entre realizar sumas, restas o multiplicaciones, mientras que la división y el módulo siempre tienen una penalización notoria, por lo que debe tenerse en cuenta.
	\item \textbf{Las operaciones en coma flotante tienen un coste pequeño}: un mito muy extendido es que las operaciones en coma flotante son muy costosas. Como podemos ver en las tablas [\ref{tab:float32}] y [\ref{tab:float64}], este no tiene por qué ser el caso, ya que todas las operaciones en coma flotante para sistemas de 64 bits resultan entre un 20\% y 40\% más rápidas que la suma de dos enteros. Este no es un mito sin fundamento, no obstante, ya que si observamos el coste de estas operaciones para sistemas de 32 bits, es muy distinto. Por tanto, nuevamente, la respuesta está condicionada por los detalles de implementación y el set de instrucciones utilizado. Si nuestra máquina no es capaz de ejecutar instrucciones SIMD, entonces es probable que las operaciones en coma flotante tengan un coste elevado.
	\item \textbf{Toda llamada a una función matemática tiene un coste elevado}: nuevamente, el coste de ejecución depende de muchas variables, principalmente de los detalles de implementación, pero como podemos ver en la tabla [\ref{tab:functions}], mientras que el coste de llamar a una función vacía es equiparable al coste de realizar una suma, algunas funciones pueden llegar a tener un coste más de 50 veces superior. Esto es un claro indicador para nuestro sistema de que realizar operaciones matemáticas complejas tiene un alto coste, y por tanto, hay también espacio para optimización, sustituyendo las funciones matemáticas dadas por otras propias, sacrificando así precisión por velocidad.
	\item \textbf{La memoria debe ser accedida de la forma más lineal posible}: acceder a memoria no es gratis. Cuando la CPU accede a memoria, carga en una caché una porción de memoria, de forma que pueda acceder a la misma de forma rápida y (casi) directa. Cuando solicitamos un recurso de memoria a la CPU, si accedemos de forma lineal, iremos accediendo continuamente a la caché de la CPU, y será en el momento en el que hayamos recorrido toda la caché cuando se producirá un fallo de caché y entonces la CPU solicitará una nueva porción de memoria, que guardará en su caché. Cuando se produce un fallo de caché, la CPU tiene que cargar nuevos datos en la caché antes de poder acceder a ellos, y este proceso lleva tiempo. Si accedemos a la memoria de forma lineal, minimizamos los fallos de caché, pues sólo se producirá uno cuando hayamos operado sobre toda la memoria que hay en caché. Sin embargo, si accedemos a memoria de forma totalmente aleatoria o inconexa, se producirá un fallo de caché por acceso a memoria, lo que ocasionará tener que cargar cada vez un bloque de memoria completo para realizar una operación únicamente sobre unos pocos bytes de la caché. Tener en cuenta por tanto, que el acceso sea lo más lineal posible, es muy importante, como podemos ver en la tabla [\ref{tab:alignment}], donde podemos ver como los accesos a memoria saltando de 100 en 100 elementos son un 150\% más lentos que accediendo a esos mismos elementos de forma lineal.
	\item \textbf{El uso de punteros no es gratuito, la indirección tiene coste}: en la tabla [\ref{tab:alignment}] podemos ver, de forma bastante consistente en apariencia, como el acceso indirecto a memoria es un 30\% más lento. En otras palabras, el uso de variables mediante punteros es un 30\% más lento que el uso de variables mediante valor. Esto tiene sentido, pues al acceder mediante un puntero se ejecuta una instrucción extra, una dereferencia. En otras palabras, mientras que para acceder a una variable de forma normal tan sólo tenemos que obtener el valor que se encuentra en la dirección de memoria de la variable, para acceder a un valor a través de un puntero debemos acceder a una variable que contiene el valor de la dirección de memoria a la que debemos acceder para obtener el valor que queremos. Es por tanto una doble referencia, por tanto una instrucción extra y por tanto, un coste añadido.
\end{itemize}

\section{Conclusiones}

¿Significa por tanto que no debemos usar punteros porque son un 30\% más lentos? ¿Que si nuestro programa accede constantemente a direcciones de memoria situadas en distintos puntos de la memoria interna, nuestro programa va a ser el doble de lento? ¿Que usar cosenos va a realentizar más de 50 veces la velocidad de nuestros programas?\\

Sí. Y no.\\

Por supuesto, el uso de punteros, el acceso no lineal a memoria o el uso de funciones matemáticas va a ralentizar nuestros programas, y no hay nada que se pueda hacer para evitarlo. La cuestión no es saber si deben utilizarse ciertas cosas, la cuestión es \textbf{cuándo}.\\

Si tenemos un programa en un sistema embebido de potencia humilde, probablemente el uso excesivo de operaciones en coma flotante podrá ser motivo suficiente como para ralentizarlo. Sin embargo, en nuestro sistema, donde disponemos de SSE, resultan hasta más rápidas que una suma normal y corriente.\\

Y por supuesto, el uso de punteros es extremadamente útil y necesario, no se trata de evitar un puntero a toda costa por ese 30\% de ganancia en rendimiento. Se trata de ser consciente que no se debe usar un puntero por usarlo, pues un uso sin causa justificada sólo implica que el código sea más lento.\\

Este set de conclusiones que se ha obtenido a partir de los tests es, como ya se ha dicho anteriormente, dependiente del sistema bajo el que se han ejecutado los mismos, y no deberían extrapolarse de forma universal bajo ninguna circunstancia.\\

Lo importante, pues, es poder identificar qué métodos o prácticas conllevan un coste considerable en nuestro sistema y cuáles no, y ser capaces de discernir con criterio cuándo conviene utilizar cada herramienta.

\section{Posibles mejoras}

Como se ha comentado previamente, haber creado nuestro propio entorno para ejecutar tests de rendimiento presenta tanto ventajas como desventajas. Es por ello que los tests desarrollados tienen validez en el entorno de desarrollo enmarcado, pero tienen una serie de limitaciones:

\begin{itemize}
	\item \textbf{Los tests han sido ejecutados bajo un solo hardware}: esto implica que los resultados obtenidos no pueden aplicarse de forma universal, si no tan solo para el hardware (y software) bajo el que han sido ejecutados. Los tests de rendimiento no eran el foco principal de este proyecto, y haber invertido tiempo en contrastar resultados en otras máquinas hubiera requerido una inversión demasiado grande. Su objetivo principal, que era otorgar una mayor comprensión de cómo funciona la maquina bajo la que se ejecutan, no obstante, han sido cumplidos con creces.
	\item \textbf{Los tests han sido ejecutados solo bajo dos sistemas operativos distintos}: de forma paralela a lo explicado para el hardware, se aplica también al software, de forma que estos resultados no se pueden tomar como universales.
	\item \textbf{Los tests son dependientes del sistema operativo}: el resultado de tiempo obtenido de los tests está condicionado por el sistema operativo, y cómo el mismo ha distribuido los recursos y los tiempos para la ejecución del programa. En otras palabras, los tests son \emph{subjetivos} al sistema operativo, pues dependen de los tiempos del sistema.
	\item \textbf{Los tests son dependientes del compilador}: los tests son dependientes del compilador utilizado y las opciones de optimización especificadas. En nuestro caso, todos los tests han sido ejecutados sin optimización alguna, por lo que los resultados pueden variar al compilar con otras opciones o compiladores.
\end{itemize}

¿Hace todo lo listado anteriormente inválidos los tests realizados?\\

En absoluto. Pero hace los resultados subjetivos. En otras palabras, los resultados obtenidos en esta máquina no deberían ser extrapolados de forma directa a cualquier otra (aunque puedan coincidir en muchos casos, no deben ser asumibles a cualquier sistema). Para tener unos tests más objetivos se podrían seguir varias estrategias:

\begin{itemize}
	\item \textbf{Distintos compiladores}: ejecutar un mismo set de tests en distintos compiladores y bajo distintas opciones de optimización (en modo depuración, sin optimizaciones y con un nivel de optimización, como mínimo).
	\item \textbf{Distintos sistemas operativos}: ejecutar un mismo set de tests bajo los principales sistemas operativos y en diversas versiones o distribuciones de los mismos.
	\item \textbf{Distintas arquitecturas}: ejecutar los tests bajo distintas arquitecturas de procesador (x86, x64). Actualmente los tests se han compilado para dos arquitecturas distintas en Windows, pero ejecutado bajo la misma.
	\item \textbf{Usar parámetros de medición cuantificables, independientes del sistema operativo}: el tiempo de ejecución, como ya hemos visto, depende en gran medida del sistema operativo y de su sistema de tareas. Basarse en otras medidas cuantificables y que sean independientes del sistema, como por ejemplo cantidad de instrucciones, pueden ayudar a obtener resultados más fácilmente cuantificables o extrapolables.
\end{itemize} 

Con todo lo dicho, llevar a cabo todas estas mediciones en si mismas puede conformar un estudio completo, ya no hablemos de la complejidad de crear un sistema capaz de llevar a cabo todas estas mediciones.\\

Es por ello que se ha mantenido un ámbito más reducido y humilde para los resultados que se presentan en este trabajo, pues si bien estos resultados nos servirán de gran ayuda en el desempeño de las demos y optimizaciones para las mismas, un análisis de los tests en mayor detalle o profundidad está fuera del ámbito de este estudio.