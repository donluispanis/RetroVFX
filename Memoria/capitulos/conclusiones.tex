%%%%%%%%%%%%%%%%%%%%%%%%%%%%%%%%%%%%%%%%%%%%%%%%%%%%%%%%%%%%%%%%%%%%%%%%
% Plantilla TFG/TFM
% Escuela Politécnica Superior de la Universidad de Alicante
% Realizado por: Jose Manuel Requena Plens
% Contacto: info@jmrplens.com / Telegram:@jmrplens
%%%%%%%%%%%%%%%%%%%%%%%%%%%%%%%%%%%%%%%%%%%%%%%%%%%%%%%%%%%%%%%%%%%%%%%%

\chapter{Conclusiones}

Este trabajo empezó por la investigación de la subcultura informática de la \emph{demoscene}, una cultura estrechamente relacionada con los gráficos por computador y cuyo mayor objetivo era la competición y superarse a sí mismo. Si bien en la actualidad se está perdiendo, en parte debido a aquellas opiniones que abogan que el bajo nivel debería ser olvidado hoy en día y en parte debido al hecho de que la \emph{demoscene} siempre ha sido una cultura muy cerrada que no ha logrado adaptarse a la masificación de la informática, lo cierto es que hay mucho que aprender de la misma. Al fin y al cabo, la esencia de la \emph{demoscene} es la pasión por el conocimiento, por superarse, por explorar los límites de un sistema y por causar un impacto en la comunidad.\\

Otro de los grandes objetivos de este trabajo era intentar implementar cuanto más efectos posibles desde cero, partiendo de un razonamiento inicial y formalizándolo hasta lograr un código de producción propia. Esto a veces podría ser considerado como reinventar la rueda. Al fin y al cabo, lo que muchas personas se preguntan es "¿para qué resolver un problema que ya ha resuelto otra persona?". Para mí, después de este trabajo, la respuesta es clara: para aprender y para descubrir nuevos puntos de vista. \textbf{Para avanzar}.\\

Y es que creo que este es un punto muy importante, \textbf{avanzar no significa olvidar el pasado}. Ante la tendencia creciente hacia las nuevas tecnologías y la abstracción, parece que hemos olvidado que todas esas tecnologías y capas de abstracción están construidas sobre los mismos principios que ya asentó Alan Turing\footnote{\url{https://en.wikipedia.org/wiki/Alan_Turing}} hace más de 80 años.\\

Por ello, pienso que es importante desarrollar un pensamiento crítico. No siempre va a resultar útil "reinventar la rueda" y habrá ocasiones en que ahondar en los detalles de implementación de ciertas librerías o programas informáticos resulte impráctico o inviable. Pero en muchos otros casos, ir a los detalles de implementación permite una mejor comprensión del funcionamiento del software y del hardware con el que se interactúa. Y conocer en profundidad el medio con el que se interactúa es vital en el transcurso de la vida profesional del programador. Del mismo modo que un médico debe conocer en profundidad el cuerpo humano, para tener capacidad de resolución ante distintos tipos de pacientes y enfermedades, lo mismo se aplica al programador en su ámbito. Conocer cómo funciona el sistema con el que se interactúa permite programar de un modo que sea más amigable de cara al sistema, permite identificar con mayor facilidad posibles \emph{bugs} y errores, permite desarrollar con mayor facilidad mejoras o refinamientos en el software y dota de una comprensión de alto nivel basada en el conocimiento del bajo nivel.\\

Ante todas las voces que abogan hoy en día que el programador debería preocuparse únicamente por resolver problemas, con independencia de la plataforma, vuelvo al paralelismo con el médico hecho anteriormente. ¿Es mejor un médico que resuelve problemas en función del \emph{modelo general} de paciente que aquel que se ciñe a los específicos del paciente que está tratando?\\

Es por ello que este trabajo ha habilitado una mejor comprensión del bajo nivel y de los gráficos por computador, que se refleja a su vez, en una mayor comprensión de los gráficos por computador y del funcionamiento de un ordenador desde el alto nivel. Muchas de las demos que se han realizado en este trabajo se han hecho sin tener en cuenta los ejemplos de implementación que se podían encontrar en la red, y si bien en algunos casos las implementaciones son similares, en otros las implementaciones varían, mostrando modelos de pensamiento distintos que dan un resultado muy similar. Se resuelve por tanto el mismo problema desde un punto de vista muy distinto. Un ejemplo claro de esto es la demo del túnel de puntos [\ref{sec:dottunnel}], que en el mundo de la \emph{demoscene} se solía implementar proyectando un túnel de puntos en el espacio tridimensional a la pantalla, mientras que en este trabajo se ha optado por imitar el efecto de la proyección, construyendo un túnel en todo momento bidimensional, a base de círculos, pero cuyo efecto final es muy similar.\\

Además, de cara a muchas demos, había más de un camino de partida inicial, y tomar una decisión u otra se ha basado a partes iguales en buscar el mejor rendimiento y el mejor resultado visual.\\

Por tanto, este trabajo refleja la importancia del pensamiento crítico a la hora de resolver un problema de programación y cómo se puede tener en cuenta el conocimiento del software y el hardware para optimizar o mejorar los resultados obtenidos.\\

No siempre será necesario tener un conocimiento en profundidad del hardware, no siempre será necesario tener un conocimiento en profundidad del software o de cada capa de abstracción. Esto sería impráctico en la mayor parte de los casos. Pero es necesario estar preparados y tener la capacidad de poder entender y profundizar en el comportamiento de un ordenador, pues esto amplía nuestro conocimiento sobre el problema a resolver y nos permite llegar a puntos de vista y conclusiones de otro modo imposibles de alcanzar.\\

Al fin y al cabo, \textbf{es muy difícil construir una buena casa sin tener en cuenta los cimientos}.