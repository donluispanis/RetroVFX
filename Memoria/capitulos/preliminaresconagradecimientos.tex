%%%%%%%%%%%%%%%%%%%%%%%%%%%%%%%%%%%%%%%%%%%%%%%%%%%%%%%%%%%%%%%%%%%%%%%%
% Plantilla TFG/TFM
% Escuela Politécnica Superior de la Universidad de Alicante
% Realizado por: Jose Manuel Requena Plens
% Contacto: info@jmrplens.com / Telegram:@jmrplens
%%%%%%%%%%%%%%%%%%%%%%%%%%%%%%%%%%%%%%%%%%%%%%%%%%%%%%%%%%%%%%%%%%%%%%%%

\chapter*{Resumen}
\thispagestyle{empty}
Me dejo el resumen para el final

\chapter*{Abstract}
\thispagestyle{empty}
Some day I will write something in here, hopefuly...

\chapter*{Motivación y objetivo general}

\thispagestyle{empty}
\vspace{1cm}

Hablando con un compañero del trabajo sobre el lenguaje ensamblador,
yo estaba intentando argumentarle su utilidad, a lo que él me contestó 
\emph{"Hoy en día, saber ensamblador es como saber latín"}.
Su afirmación zanjó el tema, pues a partir de ese momento no tuve ganas de
seguir discutiendo, pero me hizo reflexionar. ¿Es inútil el ensamblador?
¿Sirve para algo el bajo nivel?\\

Para mí, la pregunta \emph{"¿Para qué sirve saber ensamblador?"} es perfectamente equiparable a la pregunta "\emph{¿Para qué le sirve a un arquitecto conocer las herramientas y materiales con los que va a construir una casa?"}.\\

Si una edificación cayera por una mala elección del material de los cimientos por parte del arquitecto, no habría duda en a quién culpar. Nadie abogaría que la culpa no es del arquitecto porque no es su responsabilidad conocer las bases.
Sin embargo, hoy en día hay una enorme tendencia en el mundo del desarrollo software por menospreciar o infravalorar los cimientos de la programación, considerándolo algo arcaico y de carácter puramente didáctico, pero no
práctico.\\ 

Yo me opongo radicalmente a esta visión, no sólo porque estoy convencido de la importancia de conocer el bajo nivel, si no que también encuentro cierta belleza en él. Cómo instrucciones en apariencia tan simples pueden construir sistemas tan complejos. A ello, se suma una gran curiosidad por saber cómo las cosas están hechas, desde el principio.\\

Una de las cosas que encuentro más apasionantes de la computación es la capacidad de los ordenadores, máquinas inertes y carentes de inteligencia real -por el momento- para reproducir nuestra realidad a partir de modelos matemáticos.\\

Los gráficos por computador son, por lo general, complejos. Sin embargo, hoy en día es posible crear con un ordenador imágenes que parecen fotografías y son capaces de engañar al ojo humano.\\

El objetivo principal de este trabajo es ir a las raíces, y revisar algunas de las técnicas que se usaban en los orígenes de los gráficos por computador para, a partir de operaciones con bajo coste computacional, generar escenas complejas.

\cleardoublepage %salta a nueva página impar
%% Aquí va la dedicatoria si la hubiese. Si no, comentar la(s) linea(s) siguientes
%\chapter*{Agradecimientos\footnote{Por si alguien tiene curiosidad, este ``simpático'' agradecimiento está tomado de la ``Tesis de Lydia Chalmers'' basada en el universo del programa de televisión Buffy, la Cazadora de Vampiros.http://www.buffy-cazavampiros.com/Spiketesis/tesis.inicio.htm}
%}
%
%\thispagestyle{empty}
%\vspace{1cm}
%
%Este trabajo no habría sido posible sin el apoyo y el estímulo de mi colega y amigo, Doctor Rudolf Fliesning,  bajo cuya supervisión escogí este tema y comencé la tesis. Sr. Quentin Travers, mi consejero en las etapas finales del trabajo, también ha sido generosamente servicial, y me ha ayudado de numerosos modos, incluyendo el resumen del contenido de los documentos que no estaban disponibles para mi examen, y en particular por permitirme leer, en cuanto estuvieron  disponibles, las copias de los  recientes extractos de los diarios de campaña del Vigilante Rupert Giles y la actual Cazadora la señorita Buffy Summers, que se encontraron con William the Bloody en 1998, y por facilitarme el pleno acceso  a los diarios de anteriores Vigilantes relevantes a la carrera de William the Bloody.
%
%También me gustaría agradecerle al Consejo la concesión de Wyndham-Pryce como Compañero, el cual me ha apoyado durante mis dos años de investigación, y la concesión de dos subvenciones de viajes, una para estudiar documentos en los Archivos de Vigilantes sellados en Munich, y otra para la investigación en campaña en Praga. Me gustaría agradecer a Sr. Travers, otra vez, por facilitarme  la acreditación  de seguridad para el trabajo en los Archivos de Munich, y al Doctor Fliesning por su apoyo colegial y ayuda en ambos viajes de investigación.
%
%No puedo terminar sin agradecer a mi familia, en cuyo estímulo constante y amor he confiado a lo largo de mis años en la Academia. Estoy agradecida también a los ejemplos de mis  difuntos hermano, Desmond Chalmers, Vigilante en Entrenamiento, y padre, Albert Chalmers, Vigilante. Su coraje resuelto y convicción siempre me inspirarán, y espero seguir, a mi propio y pequeño modo, la noble misión por la que dieron sus vidas. 
%
%Es a ellos a quien dedico este trabajo.
%
%\cleardoublepage %salta a nueva página impar
%% Aquí va la dedicatoria si la hubiese. Si no, comentar la(s) linea(s) siguientes


\chapter*{}
\setlength{\leftmargin}{0.5\textwidth}
\setlength{\parsep}{0cm}
\addtolength{\topsep}{0.5cm}
\begin{flushright}
\small\em{
A mis padres, por estar ahí, siempre.\\
A mi hermana, por ser mi incordio y mi alegría.\\
A mi familia y amigos, por apoyarme y alegrarme los días.\\
\bigskip
A mi tutor,\\
por la visión que me ha dado sobre el mundo de la programación\\
y que tan valiosa es.\\
\bigskip
A Lola, por haber marcado la dirección cuando estábamos perdidos
}
\end{flushright}


\cleardoublepage %salta a nueva página impar
% Aquí va la cita célebre si la hubiese. Si no, comentar la(s) linea(s) siguientes
\chapter*{}
\setlength{\leftmargin}{0.5\textwidth}
\setlength{\parsep}{0cm}
\addtolength{\topsep}{0.5cm}
\begin{flushright}
\small\em{
If you give people\\
the choice of writing\\
good code or fast code,\\
there's something wrong.\\
Good code should be fast\\
}
\end{flushright}
\begin{flushright}
\small{
Bjarne Stroustrup
}
\end{flushright}
\setlength{\leftmargin}{0.5\textwidth}
\setlength{\parsep}{0cm}
\addtolength{\topsep}{0.5cm}
\begin{flushright}
\small\em{
When the whole world is silent,\\
even one voice becomes powerful.
}
\end{flushright}
\begin{flushright}
\small{
Malala Yousafzai
}
\end{flushright}
\cleardoublepage %salta a nueva página impar
