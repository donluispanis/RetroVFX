%%%%%%%%%%%%%%%%%%%%%%%%%%%%%%%%%%%%%%%%%%%%%%%%%%%%%%%%%%%%%%%%%%%%%%%%
% Plantilla TFG/TFM
% Escuela Politécnica Superior de la Universidad de Alicante
% Realizado por: Jose Manuel Requena Plens
% Contacto: info@jmrplens.com / Telegram:@jmrplens
%%%%%%%%%%%%%%%%%%%%%%%%%%%%%%%%%%%%%%%%%%%%%%%%%%%%%%%%%%%%%%%%%%%%%%%%

\chapter{Tests de rendimiento}

\section{Implementación}
\section{Resultados}

\chapter{El motor gráfico}

Antes siquiera de poder empezar a desarrollar la primera demo, es necesario crear un entorno que sea capaz de automatizar las tareas más básicas que no son competencia directa de la demo, como por ejemplo gestión de la ventana y de las entradas de teclado. Este código será común y necesario a todas las demos, pues independientemente de sus características concretas, todas ellas necesitarán una ventana y un espacio en el que poder volcar datos.\\

Es por ello que antes de empezar con la primera demo, se hizo necesario el desarrollo de un pequeño \emph{framework} que permitiese gestionar de la forma más rápida y sencilla posible aquellas tareas que no debían ser responsabilidad directa de la demo. 

\section{Investigación inicial}

Una de las principales influencias en el desarrollo de la versión inicial del motor gráfico fue OneLoneCoder\footnote{\url{https://www.youtube.com/channel/UC-yuWVUplUJZvieEligKBkA}}. Este programador británico tiene una colección de tutoriales con alto valor educativo y en muchos de ellos explica incluso técnicas de programación de la vieja escuela. Fue a raíz de visualizar estos vídeos donde vi expuestos muchos de los problemas a los que me tendría que enfrentar en el futuro.\\

El ejemplo más claro: en sus primeros vídeos, este programador siempre repite el mismo código para poner en marcha una consola usable, hasta que decide crear un modelo básico que le permita reutilizar este código.\\

Este canal ha tenido un gran valor formativo para mí, ya que me permitió identificar una serie de problemas que de otro modo sólo hubieran aparecido en un momento más avanzado del desarrollo, y que sin embargo, hubieran resultado costosos de solventar.\\

Inicialmente, estos fueron los objetivos que pretendía cubrir el motor gráfico:
\begin{itemize}
	\item Reutilización de código: tareas como abrir y cerrar la ventana o gestionar el dibujado son necesarias en absolutamente todas las demos, por lo que todo código relacionado con la ventana y/o el dibujado debería poder ser reutilizado sin tener que duplicarse.
	\item Encapsulación de toda lógica no relacionada con la demo: uno de los principales objetivos que se persiguen con la creación de un motor gráfico es la claridad. La implementación de una demo \textbf{sólo debe contener lógica que está directamente relacionada con sus detalles de implementación}, es decir, con los algoritmos o técnicas de los que la demo hace uso. De este modo, el código de una demo sólo refleja la lógica de la misma, sin exponer la lógica necesaria para la de gestión de ventana, que no es responsabilidad de la misma. Esto permite un código más claro y conciso, más fácil de implementar, usar, refinar y entender.
		\begin{itemize}
			\item Encapsulación de la ventana: una demo no debe ser consciente de qué es necesario para crear o borrar una ventana, todo lo que debe hacer es ser poder decir "quiero crear una ventana" o "quiero cerrar la ventana" pero no debe ocuparse de los detalles de implementación.
			\item Encapsulación del dibujado: una demo no debe tener responsabilidad de gestionar el dibujado en ventana. Todo lo que una demo necesita saber es en qué lugar de memoria debe escribir para que esos datos sean dibujados en pantalla, pero no debe encargarse de la gestión del dibujado.
		\end{itemize}
	\item Abstracción de la plataforma: el código de una demo no debe contener detalles de implementación relativos a la plataforma en que se ejecuta. Desde el punto de vista de la demo, todo lo que importa es el algoritmo, y este debe ser el mismo independientemente del sistema operativo y del \emph{hardware} sobre el que se ejecuta.
\end{itemize}

Durante el desarrollo, no obstante, nuevas necesidades se irían añadiendo, ya fuera por nuevas decisiones de diseño, refinamiento de código o por nuevas necesidades de las demos:
\begin{itemize}
	\item Abstracción de las librerías y tecnologías utilizadas: tras varias iteraciones sobre el desarrollo inicial, fue necesario un refinamiento. El motor gráfico contenía demasiada lógica, y era lógica acoplada a la gestión de la ventana o del dibujado. Esto levantó una pregunta: ¿y si en algún momento necesito cambiar las librerías que utilizo o incluso prescindir de las mismas? Esta era una posibilidad bastante probable, dado que a lo largo del desarrollo de un proyecto y conforme surgen nuevas necesidades, puede que las tecnologías elegidas inicialmente no satisfagan las condiciones actuales. Por tanto, el motor gráfico no debía estar acoplado a las tecnologías que usaba, si no que debía mediar con ellas mediante el uso de interfaces.
	\item Abstracción de los eventos de teclado: conforme el desarrollo avanzó, se hizo aparente que en muchas ocasiones era útil permitir al usuario modificar parámetros de la demo en tiempo real, en cierto modo permitir "jugar" con la demo. Era necesario por tanto permitir el manejo de eventos de teclado, aunque su uso debía estar abstraído de su implementación, de forma que desde el punto de vista de la demo, todo lo que se pudiera hacer es "quiero saber el estado de esta tecla".
	\item Abstracción del dibujado de texto en pantalla: una vez más, al continuar con el desarrollo, se hizo aparente la necesidad de poder dibujar texto en pantalla. El motor gráfico debía ser por tanto capaz de abstraer o enmascarar las rutinas de dibujado del texto, de modo que desde la perspectiva de la demo todo lo que importase fuera dibujar un texto con un color, posición y tamaño determinados, independientemente de la implementación.
	\item Uso de mecanismos de dibujado seguros para formas básicas: aunque inicialmente parecía que cualquier tipo de dibujado debía ser responsabilidad de la demo, pronto se hizo aparente que ciertas rutinas se repetían de forma constante. Además, mientras que en un inicio el dibujado de un punto o una línea era responsabilidad de la demo, pronto se vio que desde el punto de vista de la demo, estas responsabilidades no tienen interés, ya que la capacidad de poder dibujar una línea es importante, pero no cómo se dibuja. 
		\begin{itemize}
			\item Dibujado de puntos: desde la perspectiva de una demo, tan sólo importan la posición, color y tamaño de un punto que se quiera dibujar en pantalla. La gestión de si ese punto está dentro o fuera de los límites de pantalla o la gestión del tamaño del punto no debería ser competencia de la demo, si no del motor.
			\item Dibujado de líneas: una demo debe ser capaz de solicitar el dibujado de una línea dados dos puntos, un color y un tamaño, pero no debe responsabilizarse de la gestión de los límites en pantalla ni del algoritmo de dibujado para una línea.
			\item Dibujado de rectángulos: una demo debe ser capaz de dibujar rectángulos en pantalla, especialmente útiles para el borrado de la pantalla o de regiones de la misma, pero no debe conocer sus detalles de implementación.
		\end{itemize}
\end{itemize}

Con todos estos puntos en mente, y de forma progresiva, se fue desarrollando, revisando y refinando la creación de un motor gráfico que sirviera como marco de trabajo efectivo para el desarrollo de una demo.

\section{Características}
\subsection{Detectar input}
\subsection{Dibujar texto}
\subsection{Dibujar líneas}

\chapter{Demos clásicas}

\section{Fuego}

\subsection{Investigación inicial}
\subsection{Planteamiento formal}
\subsection{Implementación}
\subsection{Refinamiento}
\subsection{Resultado}

\section{Geometría}

\subsection{Investigación inicial}
\subsection{Planteamiento formal}
\subsection{Implementación}
\subsection{Refinamiento}
\subsection{Resultado}

\section{Planos infinitos}

\subsection{Investigación inicial}
\subsection{Planteamiento formal}
\subsection{Implementación}
\subsection{Refinamiento}
\subsection{Resultado}

\section{Plasma}

\subsection{Investigación inicial}
\subsection{Planteamiento formal}
\subsection{Implementación}
\subsection{Refinamiento}
\subsection{Resultado}

\section{RotoZoom}

\subsection{Investigación inicial}
\subsection{Planteamiento formal}
\subsection{Implementación}
\subsection{Refinamiento}
\subsection{Resultado}

\section{Deformaciones de imagen}

\subsection{Investigación inicial}
\subsection{Planteamiento formal}
\subsection{Implementación}
\subsection{Refinamiento}
\subsection{Resultado}

\section{Túnel de puntos}

\subsection{Investigación inicial}
\subsection{Planteamiento formal}
\subsection{Implementación}
\subsection{Refinamiento}
\subsection{Resultado}

\chapter{Demo final}

\subsection{Investigación inicial}
\subsection{Planteamiento formal}
\subsection{Implementación}
\subsection{Refinamiento}
\subsection{Resultado}
